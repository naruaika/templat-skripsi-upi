\newacronym{2d}{2D}{dua dimensi}
\newacronym{gcn}{GCN}{Gabor Convolutional Network}
\newacronym{ann}{ANN}{Artificial Neural Network}
\newacronym{cnn}{CNN}{Convolutional Neural Network}
\newacronym{dgltp}{D-GLTP}{Directional Gradient Local Ternary Patterns}
\newacronym{gltp}{GLTP}{Gradient Local Ternary Patterns}
\newacronym{eeg}{EEG}{electroencephalography}
\newacronym{lstm}{LSTM}{Long Short-Term Memory}
\newacronym{dhcnn}{DHCNN}{Deep Hybrid Convolutional Neural Networks}
\newacronym{lmp}{LMP}{Local Motion Patterns}
\newacronym{facs}{FACS}{Facial Action Coding System}
\newacronym{femg}{fEMG}{Facial Electromyography}
\newacronym{ml}{ML}{Machine Learning}
\newacronym{frs}{FRS}{Facial Region Segmentation}
\newacronym{dcnn}{DCNN}{Deep Convolutional Neural Network}
\newacronym{gofs}{GoFs}{Gabor Orientation Filters}
\newacronym{roi}{RoI}{region of interest}
\newacronym{adam}{Adam}{Adaptive Momentum}
\newacronym{ti}{TI}{Teknologi Informasi}
\newacronym{ser}{SER}{Speech Emotion Recognition}
\newacronym{cpu}{CPU}{Central Processing Unit}
\newacronym{ram}{RAM}{Random Access Memory}
\newacronym{gpu}{GPU}{Graphical Processing Unit}
\newacronym{os}{OS}{Operating System}

\newglossaryentry{komputasiafektif}{
    name={komputasi afektif},
    description={Studi multidisipliner pada pengembangan sistem yang mampu mengenali, memahami dan menghasilkan emosi manusia}
}
\newglossaryentry{komunikasiverbal}{
    name={komunikasi verbal},
    description={Komunikasi yang melibatkan kata-kata, baik lisan maupun tulisan}
}
\newglossaryentry{komunikasinonverbal}{
    name={komunikasi nonverbal},
    description={Komunikasi yang melibatkan perilaku fisik (bahasa tubuh) termasuk ekspresi wajah, postur tubuh dan isyarat}
}
\newglossaryentry{sixbasicemotions}{
    name={\emph{six basic emotions}},
    description={Penggolongan emosi yang dianggap universal. Terdiri dari marah, takut, senang, sedih, terkejut, dan merasa jijik \shortcite{ekman1970universal}.}
}
