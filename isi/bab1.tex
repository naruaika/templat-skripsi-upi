\chapter{Pendahuluan}

\section{Latar Belakang}
Pengenalan emosi manusia sangat bermanfaat di sektor-sektor terkait \gls{komputasiafektif}\footnote{Komputasi afektif adalah studi multidisipliner pada pengembangan sistem yang mampu mengenali, memahami dan menghasilkan emosi manusia.}. Umpamanya di periklanan, pengenalan emosi dapat diterapkan berdasarkan hubungan berbanding lurus antara kualitas respons emosional masyarakat terhadap iklan dengan peningkatan penjualan \shortcite{sujata2018facial}. \dots

\section{Identifikasi Masalah}
Dari latar belakang di atas, kami mengidentifikasi beberapa permasalahan spesifik yang diangkat di penelitian ini sebagai berikut.
\begin{enumerate}
    \item Rancangan pemodelan pengenalan emosi melalui ekspresi wajah menggunakan \acrshort{gcn} melalui pendekatan \emph{facial region segmentation}.
    \item \dots
\end{enumerate}

\section{Rumusan Masalah}
Untuk memperjelas permasalahan yang akan diteliti, maka kami merumuskan permasalahan tersebut sebagai berikut.
\begin{enumerate}
    \item Bagaimana rancangan pemodelan pengenalan emosi melalui ekspresi wajah menggunakan \acrshort{gcn} melalui pendekatan \emph{facial region segmentation}?
    \item \dots
\end{enumerate}

\section{Tujuan Penelitian}
Berdasarkan rumusan di atas, tujuan yang hendak dicapai dari penelitian ini terangkum ke dalam poin-poin berikut ini.
\begin{enumerate}
    \item Merancangan pemodelan pengenalan emosi melalui ekspresi wajah menggunakan \acrshort{gcn} melalui pendekatan \emph{facial region segmentation}.
    \item \dots
\end{enumerate}

\section{Batasan Penelitian}
Agar penelitian dapat dilaksanakan sesuai dengan bidang fokus permasalahan yang telah ditentukan, maka berikut ini adalah batasan penelitian ini.
\begin{enumerate}
    \item Diasumsikan bahwa ekspresi wajah adalah representasi paling akurat dari emosi manusia.
    \item \dots
\end{enumerate}

\section{Manfaat Penelitian}
Hasil dari eksperimen ini diharapkan dapat memberikan manfaat-manfaat berikut ini.
\begin{enumerate}
    \item Memberikan pengetahuan dan pengalaman pertama bagi kami di penelitian pengenalan emosi otomatis melalui ekspresi wajah.
    \item \dots
\end{enumerate}

\section{Struktur Organisasi Skripsi}
Struktur organisasi skripsi berisi perincian tentang urutan skripsi yang terorganisasi ke dalam lima bab, mulai dari bab I hingga bab V. \dots
